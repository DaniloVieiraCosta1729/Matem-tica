\documentclass[12pt]{article}

% Pacotes para configurações adicionais
\usepackage[margin=1in]{geometry} % Configuração das margens
\usepackage{times} % Usar a fonte Times New Roman
\usepackage{setspace} % Espaçamento entre linhas
\usepackage{sectsty} % Formatação de títulos de seção
\usepackage{graphicx} % Suporte para inclusão de figuras
\usepackage{amsmath} % Pacote para equações matemáticas
\usepackage{amsfonts}

% Configurações adicionais
\renewcommand{\familydefault}{\rmdefault} % Definir a fonte padrão como Times New Roman
\linespread{1.5} % Espaçamento de 1.5 entre linhas
\allsectionsfont{\sffamily\bfseries} % Títulos de seção em negrito e sem serifa

\title{Espaços Vetoriais}
\author{Danilo Vieira Costa}
\date{17/05/2024}

\begin{document}

\maketitle

\begin{abstract}
Vamos definir o que é um espaço vetorial, deduzir alguns fatos básicos e explorar alguns exemplos.
\end{abstract}

\section{Espaço Vetorial (ou Linear)}
Dizemos que um conjunto E é um espaço vetorial e chamamos os seus elementos de vetores se esse conjunto está munido de duas operações, $\cdot$ e $+$ de modo que essas operações possuam algumas propriedades específicas que vamos ver a seguir, porém, vamos antes definir o domínio e contra-domínio dessas operações.
$$
\begin{matrix}
    \cdot : \mathbb{R} \times E\longrightarrow E \\
    \left( \alpha , u\right) \longmapsto v
\end{matrix}
$$
$$
\begin{matrix}
    +: E \times E \longrightarrow E \\\left(u,v\right) \longmapsto w
\end{matrix}  
$$
A forma como definimos essas operações já garante o fechamento de $E$ com relação a essas operações.
As propriedades que essas operações devem possuir para que $E$ seja um espaço vetorial são:\\
Para todo $u$, $v$, $w \in E$ e  $\alpha$ e $\beta \in \mathbb{R}$
\begin{itemize}
    \item Comutatividade da soma: $u+v=v+u$;
    \item Associatividade da soma e produto: $u+\left(v+w\right)=\left(u+v\right)+w$ e $\left(\alpha \beta \right)u=\alpha\left(\beta u\right)$;
\end{itemize}
\end{document}
